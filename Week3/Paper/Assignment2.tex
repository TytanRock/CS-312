% Document Class is specified here. We're using the Homework Template
\documentclass[12pt, a4paper]{article}

\usepackage{graphicx}
\usepackage{float}
\usepackage{geometry}
\usepackage{verbatim}

\geometry{a4paper, margin=1in}

\title{Week 3 CS-312 Homework}
\author{
	Cory Ness
	\and
	Jack Engledow
	\and
	James Sgrazzutti
}

\begin{document}

\maketitle

\section{Problem 3.1}
\subsection{Question}
Give an NFA for the language of RE $a^{*}b+b^{*}a$
\subsection{Answer}

\section{Problem 3.6}
\subsection{Question}
Show how to modify an NFA to have a unique accept state with no transition ending at the start state and no transition starting at the accept state.
\subsection{Answer}

\section{Problem 3.7}
\subsection{Question}
If $M$ is a DFA accepting language $B$, then exchanging the accept and reject states gives a new DFA accepting teh complement of $B$. Does this work for an NFA? Discuss.
\subsection{Answer}

\section{Problem 3.9}
\subsection{Question}
For the following NFA, use the subset construction to produce an equivalent DFA.
\subsection{Answer}

\section{Problem 3.11}
\subsection{Question}
Provide an algorithm to tell if the language is infinite if the input is
\subsubsection{An RE}
\subsubsection{An NFA}

\section{Problem 4.1}
\subsection{Question}
Show that the set of regular languages is closed under reversal. That is, if $L$ is regular, then so is $\{x^{R} : x \in L\}$ where $x^{R}$ denotes the reversaal of string $x$.
\subsection{Answer}

\section{Problem 4.11}
\subsection{Question}
Show taht $\{x\#x : x \in \{0,1\}^{*}\}$ is nonregular. (The hash mark/pound sign is a special symbol that should only occur in the middle of the input string.)
\subsection{Answer}

\section{Problem 4.13b}
\subsection{Question}
State whether the set of binary nonpalindromes is regular or not. If not, give a proof that it is nonregular.
\subsection{Answer}

\section{Problem 4.14}
\subsection{Question}
Explain what is wrong with the following "proof" that the language $L$ of an RE $a^{*}b^{*}$ is nonregular.

\begin{quotation}
Suppose $L$ were regular. Then it would be accepted by a DFA with, say, $k$ states. Consider the string $z=0^{k}1^{k}$. Split $z=uvw$ with $v=01$. Then $uv^{2}w$ is not in $L$. This is a contradiction of the Pumping Lemma, and so our supposition is false.
\end{quotation}
\subsection{Answer}

\section{Problem 4.20}
\subsection{Question}
Convince your grandmother that there is no FA that accepts the language of binary strings with an equal number of $0$'s and $1$'s.
\subsection{Answer}


\end{document}